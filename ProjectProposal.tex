\documentclass[a4paper,10pt]{article}
\usepackage{amsmath}
\usepackage{amsthm}
\newtheorem{mydef}{Definition}
\usepackage{url, hyperref}
\usepackage{amsthm}
\usepackage{amsfonts}
\usepackage{amssymb}
\newtheorem{theorem}{Theorem}
\newtheorem{lemma}{Lemma}
\usepackage{fullpage}
\usepackage{tikz}
\usepackage{float}
\usepackage{listings}
\usepackage{color}
\definecolor{dkgreen}{rgb}{0,0.6,0}
\lstset{language=Matlab,
   keywords={break,case,catch,continue,else,elseif,end,for,function,
      global,if,otherwise,persistent,return,switch,try,while},
   basicstyle=\ttfamily,
   keywordstyle=\color{blue},
   commentstyle=\color{dkgreen},
   stringstyle=\color{magenta},
   numbers=left,
   numberstyle=\tiny\color{gray},
   stepnumber=1,
   numbersep=10pt,
   backgroundcolor=\color{white},
   tabsize=4,
   showspaces=false,
   showstringspaces=false}
\usepackage{mathtools}
\usepackage[T1]{fontenc}
\usepackage{subfigure}
\usepackage{algorithm}
\usepackage{algpseudocode}
\bibliographystyle{plain}

\title{Project Proposal: Efficient segmentation algorithms for shark fin identification}
\author{L. Cilli\'{e}, 16010450}
\date{August 1, 2013}

\begin{document}
\maketitle
\lstset{language=Matlab}

\newpage
\tableofcontents
\newpage

\section{Problem statement and motivation}  
Just as humans are identified by their fingerprints, sharks have an unique fin structure.  Identifying shark sightings these days can be very valuable.  In order to do that and also to match shark sightings, photos of shark fins, as will be shown in the proposal, will be analysed by a computer.  This will consist of comparing the edges in the image and segmenting the foreground, the shark fin being most prominent,  from the background, the sea.  Although the photos provided are of good quality and only include the fin, the foreground and background properties can still vary significantly, making this a non-trivial problem.  This project aims to investigate efficient segmentation algorithms for shark fin identification.  The approach will consist of evaluating (by means of a computer) different methods for classifying foreground and background, as well as segmenting the foreground successfully.  By doing so, the burden on biological researchers can be reduced. 


\section{Background}
Image segmentation is well known in the field of image processing.  It consists of partitioning a digital image into multiple segments for some or other reason.  Usually to make the image easier to analyse.  In other words each pixel is assigned a label and all the pixels with the same label share a certain property.  Image segmentation is typically used to locate an object or boundaries in an image.  Well known examples where image processing are used is in medical imaging to locate tumours for example face detection, face and fingerprint recognition and video surveillance.


\section{Segmentation algorithms}
\subsection{Different categories and techniques}
Segmentation algorithms can be arranged into different categories \cite{is}.  Some of the most well-known categories are
\begin{itemize}
 \item \textbf{Thresholding} \cite{is} Possibly the most basic method in the field.  It relies on selecting the best threshold value to convert a gray-scale image into a binary image.  Segmentation is then done on the binary image. 
 \item \textbf{Clustering} \cite{is} For these methods, the $K$-means algorithm is typically used to partition an image into $K$ clusters, whereafter segmentation will follow.
 \item \textbf{Compression-based} \cite{is} This method conjectures that the optimal segmentation is the one that minimizes, when considering all possible segmentations, the coding length of the data.
 \item \textbf{Histogram-based} \cite{is} This is a very efficient method in the sense that it only passes through all the pixels once.  Then a histogram of all the pixels in the image is computed.  The peaks and valleys, colour intensity can be used as a measure, are then used to locate clusters in the image.
 \item \textbf{Edge detection} \cite{is} Edge detection is a well-developed field in image processing.  Region boundaries and edges are closely related, since there are usually a sharp adjustment at region boundaries.  Edge detection techniques can therefore be used as a bases for different segmentation algorithms.  
 \item \textbf{Region growing} \cite{is}  This method takes as input not only the image, but also a set of seeds, which marks the objects to be segmented.  Thereafter neighbouring, unallocated pixels are compared to the specific seed point.  That is how the region grows iteratively.  The pixels are compared using the difference between intensity value and the region's mean.  Pixels are then allocated to a specific region if that difference is a minimum.  The process terminates if all pixels belong to a region.  
 \item \textbf{Watershed transformation} \cite{is} See description in 3.2.    
 \item \textbf{Graph partitioning} \cite{is} This method is based on modelling the image as a weighted, undirected graph, where the nodes represent the pixels and the weights represent the similarity between neighbouring pixels. Then the graph is partitioned into clusters according to a specific criterion.  Each partition is then considered a object segment in the image.       
 \item \textbf{Trainable} \cite{is} Also known as Neural network segmentation, this process involves processing the small areas of an image by means of an artificial neural network or a set of neural networks.  Thereafter, using the categories recognised by the network, the decision-making mechanism marks the image accordingly and segmentation is done.
\end{itemize}

\subsection{The Watershed algorithm}
One of the first algorithms tested was the classic watershed algorithm, as implemented in \cite{scikit}.  The algorithm starts with user-defined markers, called seed points, which can be viewed as little holes in the image, whereafter pixel values are treated as a topography/landscape.  The algorithm then floods basins from the user-defined markers until basins which attribute to different markers meet at watershed lines.  In this case marker positions are chosen as the local maxima of the image.  Thereafter the segmentation is done on the gradient image.  The result of the watershed algorithm applied to a shark fin image is shown below.

\begin{figure}[H]
\centering
\includegraphics[width=4in,height=2in]{watershed.png} 
\label{fig1}
\caption{Result using the watershed algorithm}
\end{figure}

\noindent The first image shows the gradient image of the original shark fin image.  The second image shows the pixels that are identified as a local maxima.  The third image shows how the basins flooded until watershed lines are reached.  Putting the correct segments together, which will require a lot of hard work, an effective segmentation can be done.

\subsection{The Random Walker algorithm}
The basic method (from the graph partitioning category shown above), as described in \cite{rw} and \cite{rw1}, is as follows.  The user labels a few pixels as foreground or background for example.  This is then called the seeds.  A random walker is then released from each of the unlabelled pixels.  Thereafter, the probability that a certain pixel's random walker first arrives at a seed with a label, is computed.  In other words, if the user labels $n$ pixels, each with a different label, then the probability that the random walker leaving the pixel will first arrive at a certain labelled pixel, must be computed.  The latter is done by modelling the image as a weighted graph, where the weight of an edge reflects the similarity(intensity values) between pixels, and solving a system of linear equations.  The pixel is then assigned the label of the seed for which it is most likely to reach.  The process is repeated until each pixel is assigned a specific label. 

\subsection{Random forests as a semantic segmentation algorithm}
A semantic segmentation algorithm is done by assigning category labels to a set of super pixels obtained by clustering the joint colour space and coordinate space using the mean shift algorithm \cite{ms}.
The method of interactive semantic segmentation as in \cite{RF} can be described as follows.  Each image is presented as a set of super pixels, where a super pixel is a set of neighbouring pixels.  The first image in the data set can be segmented with any appropriate segmentation algorithm.  An appearance model is now created.  The next image is now segmented automatically using the appearance model. \\  

\noindent During the segmentation the user may correct mistakes by relabelling and thus updating the appearance model.  Every time the user approves the segmentation, the system learns from the new labelling information. As image segmentation continues, user time spent on correcting labels reduces and thus the rate of image labelling increases.  \\

\noindent This method can be helpful in the sense that it works with a database of images of the same kind.  Since the shark fin images are all of the same type, it would be very beneficial to manually segment the first image only and then automatic segmentation will follow.  By updating the appearance model, one can also be certain that better segmentation results will follow.  \\  

\section{The region growing GrowCut algorithm}
\subsection{What is a cellular automaton?}
A cellular automaton consists of a grid of cells, where each one of the cells can be in a finite number of states, say on and off.  This must be specified beforehand.  Around each cell, a set of cells, called the cell's neighbourhood, is defined.  An initial state, at time t = 0, is also assigned to each cell.  A new generation of cells is then created according to a fixed rule or mathematical function that determines the new state of the cell, by looking at the current state of the cell as well as that of its neighbourhood.  This rule is then applied to all of the cells simultaneously.  In this way, the cell's state gets updated.  Note that the rules for each cell are the same and do not change over time.  Two of the most common neighbourhood systems used are the Von Neumann neighbourhood and the Moore neighbourhood which are shown below.  Probably the most well-known example of a two dimensional automaton is Conway's Game of Life.  See \cite{gol} for further details.

\begin{figure}[H]
\centering
\mbox{\subfigure{\includegraphics[width=1in]{VonNeumann.png}} \quad
\subfigure{\includegraphics[width=1in]{Moore.png}}} \caption{Von Neumann and Moore neighbourhood systems \cite{n}}
\end{figure}

\subsection{The GrowCut algorithm as implemented in Python}
The GrowCut algorithm \cite{alg} is an interactive, multi-label segmentation algorithm for N-dimensional images.  The algorithm is based on cellular automata, ie.,  the user labels a few pixels and the rest of the image is then segmented automatically by a cellular automaton. \\

\noindent The pseudo code for the automata evolution rule is shown below. 
\begin{algorithm}[H]
\begin{algorithmic}[1]
 \State // For each cell...
 \For{$\forall p \in P$}
 \State // Copy previous state
 \State $l^{t+1}_{p} = l^{t}_{p}$;
 \State $\theta_{p}^{t+1} = \theta_{p}^{t}$;
 \State // Neighbours try to attack the current cell
 \For{$\forall q \in N(p)$}
 \If{$g(\| \overrightarrow{C}_{p} - \overrightarrow{C}_{q} \|_{2}) \cdot \theta^{t}_{q} > \theta_{p}^{t}$}
 \State $l^{t+1}_{p} = l^{t}_{q}$
 \State $\theta^{t+1}_{p} = g(\| \overrightarrow{C}_{p} - \overrightarrow{C}_{q} \|_{2}) \cdot \theta^{t}_{q}$
 \EndIf
 \EndFor
 \EndFor
\end{algorithmic}
\end{algorithm}

\noindent where $g$ is a monotonous decreasing function bounded to $[0, 1]$.  The function is given by
\[
g(x) = 1 - \frac{x}{max\| \overrightarrow{C} \|_{2}}. 
\]

\noindent The cell state referred to is considered a triplet $(l_{p}, \theta_{p}, \overrightarrow{C}_{p})$, where $l_{p}$ is the label of the cell ($K$ labels in total), $\theta_{p}$ is the 'strength' of the cell and $\overrightarrow{C}_{p}$ is the cell feature vector.  Without loss of generality it can be assumed that $\theta_{p} \in [0,1]$. 
The initial state of the pixels is set to $l_{p} = 0, \theta_{p} = 0, \overrightarrow{C}_{p} = RGB_{p}$, where $RGB_{p}$ is a three dimensional vector of the pixel's colour in 
the RGB space.  The goal of the segmentation is to assign one of the possible $K$ labels to each one of the pixels.  The user starts the segmentation by marking specific pixels as
foreground and others as background.  This sets the initial state of each pixel.  While the labels are being updated by the rule above, which will be discussed in greater detail hereafter, the user can correct and guide the process if desired.  The number of iterations the algorithm has to follow can also be set beforehand.  \\

\noindent The evolution rule is as follows.  Consider cell $p$.  Now consider one of the cells in its neighbourhood, type specified beforehand, say cell $q$.  If the difference in intensity of $p$ and $q$, times the cell strength of $q$, is greater than the cell strength of $p$, the label of $p$ is updated to the label of $q$.  Whereas the cell strength of $p$ is updated to the difference in intensity of $p$ and $q$, times the cell strength of $q$.  These states are then the starting values for the next iteration. \\

\noindent An interactive implementation of the algorithm, \cite{growcut}, was received from Dr. Nathan Faggian, Senior Technical Officer at Bureau of Meteorology, Australia.  The result of his implementation, with changes to the damping function, applied to one of the shark fin images is shown below. 
\begin{figure}[H]
 \centering
 \includegraphics[width=3in, height=1.5in]{haaim}
 \caption{Original Python implementation}
 \label{fin}
\end{figure}    


\noindent Following the pseudo code given above, we developed our own implementation of the GrowCut algorithm in Python in order to aid our understanding of the algorithm.  It is now also easier to work on the optimisation, ie. speed and quality of the segmentation.  The result of our own implementation applied to one of the shark fin images, as well as a strength map of the cells, is shown below
\begin{figure}[H]
 \centering
 \includegraphics[width=4in, height=1.7in]{demo1.png}
 \caption{Our Python implementation}
\end{figure}   


\noindent Cython \cite{cython} is a language that compiles to a Python extension module.  It is a super set of the Python language that allows calling C-functions and declaring C-types on variables and class attributes.  This results in generating very efficient C-code from Cython code.  Cython was used to increase the speed of our own basic implementation of the GrowCut algorithm.  This was one of the steps to better the performance of the segmentation.       

\paragraph  It is clear that both implementations gave remarkable results and a definite advantage of this algorithm compared to others, is of course the simplicity, yet power of each step of the algorithm. 


\section{Objectives}
We aim to achieve the following.
\begin{itemize}
 \item To do an in-depth comparison between the different segmentation algorithms considered, whereafter sensible conclusions can be made. 
 \item To further study the GrowCut algorithm in order to optimise our implementation of the algorithm.  To do that, we will look at different 'attacking' strategies, investigate the condition of boundary smoothness and look at different texture features of the image.
 \item To do a further study on Random Forests as a classification method and how it can be implemented to be of use.  Also study the topic of conditional random fields which relates to the latter.
 \item Get the Random Walker segmentation algorithm running on a shark fin image.
\end{itemize}

\newpage
\bibliography{pp}

\end{document}
