\documentclass[a4paper,10pt]{article}
\usepackage{amsmath}
\usepackage{amsthm}
\newtheorem{mydef}{Definition}
\usepackage{url, hyperref}
\usepackage{amsthm}
\usepackage{amsfonts}
\usepackage{amssymb}
\newtheorem{theorem}{Theorem}
\newtheorem{lemma}{Lemma}
\usepackage{fullpage}
\usepackage{tikz}
\usepackage{float}
\usepackage{listings}
\usepackage{color}
\usepackage{listings}
\usepackage{algorithm}
\usepackage{algpseudocode}
\bibliographystyle{plain}

\title{Project Proposal Presentation: Efficient segmentation algorithms for
shark fin identification}
\author{L. Cilli\'{e}, 16010450}
\date{1 August 2013}

\begin{document}
\maketitle
\section{Motivation and problem statement}
Just as humans are identified by their fingerprints, sharks have an unique
dorsal fin structure.  \\

Because of environmental issues, estimating the
population of a specific shark species, for example, has become a hot topic
under
marine scientists. \\

The idea behind the project originated when PhD
student in Marine Biology, Ms. Sara Andreotti, approached Dr. van der Walt for
help
regarding a problem she encountered in her research. \\ 

The problem presented was
as follows.  The work Ms. Andreotti does involve going out into the ocean,
taking photographs of shark fins and identifying those sharks to later study their ecological and behavioural
patterns, for example. \\

All these photos are put into a databases. But to
make any reliable conclusions, such as: Have we seen this shark before?  How
many
times? Where? When?, there has to be some form of identification in the
databases. \\

The goal is to group photos of a certain individal together.  We
already know that it is possible to categorize each image, because of the
uniqueness of the dorsal fin.  One would just have to find a way to match new
input data with existing images in the databases.  One can only imagine the
diffuculty in doing this manually. \\

Since Ms. Andreotti's
focus is on the distribution and movements of sharks and not on developing new
software, we can make a substantial contribution by researching this topic. \\

By doing so, the burden on marine scientists can be 
reduced.
 \\

We will investigate different
methods for classifying the foreground and background of the image
and then segmenting the foreground, or certain parts of it, successfully for the
matching process.  

\section{Examples of shark fin images}
Note the unique dorsal part of the fin.  Although theses photos are of a good
quality, only including the fin, the foreground and background properties can
vary significantly as you can see here.


\section{Background -- Image segmentation}
Image segmentation is well known in the field of image processing.  The basic
principle of image segmentation is partitioning an image into multiple segments
for some given purpose, for example to locate an object or boundaries in the
image.  Each pixel is assigned a label and all pixels with
the same label share a certain property and forms a segment.  \\

Well known
examples where image processing is applied is in medical imaging to locate
tumours, face and fingerprint recognition as well as in video surveillance.
Today there are a number of different techniques to do image
segmentation of which a few will be discussed later.


\section{Algorithms from the scikit-image Python library}
Some of the algorithms we investigated were the Watershed and Random Walker
algorithms from the scikit-image library.  But we came across a much simpler,
yet more powerful algorithm called the GrowCut segmentation algorithm.  This
algorithm performed exceptionally well in comparison with the others. 

\section{The region growing GrowCut segmentation algorithm}
The GrowCut algorithm is an interactive, multi-label segmentation
algorithm for N-dimensional images.  This algorithm is based on cellular
automata, i.e.,  the user labels a few pixels and the rest of the image is then
segmented automatically by a cellular automaton. \\

An important principle of a cellular automaton
is the evolution rule, by which the state of the cell, in our case foreground or
background, is updated.  \\

In other words, after specifying a neighbourhood around the cells, in our case
pixels, each iteration consists of
the neighbouring cells 'attacking' the cell under consideration.  The state of
this cell is then updated
according to the evolution rule which is just a mathematical formula. This procedure then iterates throughout the
image until each pixel has a label, either foreground or background. \\

In collaboration with one of the Masters
students, we are building a pipeline for Ms. Andreotti
such that the input data is the shark fin image and the output is a specific
classification of that image.\\

Natuarally we would want the orientation of all the images to be the same.\\

Specifying a few foreground and background pixels for segmentation would then be easier. \\

The 
next step is to extract the unique part of the fin for use in the matching process.



\section{DARWIN -- Alternative software}
There appeared an article in Die Burger recently claiming
that the Darwin software could also be used for the identification and matching of
shark fin images to estimate the Great White shark population in the Gansbaai
area.  \\

This claim was tested by Ms. Andreotti and she found that in only
54\% of the cases the correct matching took place, making this software highly
unreliable. \\

We are therefore still motivated to produce more robust and reliable software. 

\section{Conclusions}
Future work includes adjusting the evolution rule in the GrowCut algorithm
and completing the pipeline for Ms. Andreotti.\\

For me, one of the most satisfying aspects of this project is the practical
application of the topic in the real world.  Not only this, but also the
privilege we have to present our software to Ms. Andreotti, in aid of
her PhD.   \\

Thank you very much for attending my presentation.

\end{document}