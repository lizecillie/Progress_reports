\documentclass{beamer}
\usepackage{amsthm}
\usepackage{amssymb}
\usepackage{amsmath}
\usepackage{amsfonts}
\usepackage{graphicx}
\usepackage{tikz}
\usetheme{Rochester}
\usecolortheme{seahorse}
\usefonttheme{structuresmallcapsserif}
\setbeamertemplate{caption}[numbered]
\setbeamertemplate{footline}[frame number]
\usepackage{subfigure}
\newcommand{\myitem}{\item[$-$]}

\title{Efficient segmentation algorithms for shark fin identification}
\subtitle{Honours project proposal}
\author{Lize Cilli\'{e}, Project advisors: Dr. S.J. van der Walt and Prof. B.M.
Herbst}
\date{1 August 2013}
\institute{Department of Applied Mathematics, Stellenbosch University}

\begin{document}
\maketitle


\begin{frame}
\frametitle{Motivation and problem statement}
\begin{itemize}
\item Sharks have an unique dorsal fin structure.
\item The value of analysing shark fins.
\item Motivation behind the project.
\begin{itemize}
 \myitem PhD student in Marine Biology at Stellenbosch University, Ms. Sara
Andreotti.
 \myitem Databasis of shark fin images that needed to be catogorized for studying
behavioural and ecological patterns of sharks.
\end{itemize}
\item Methodology.
\begin{itemize}
 \myitem We will investigate different methods for classifying the foreground and
background.
 \myitem Segmenting the foreground successfully for matching. 
\end{itemize}
\item Advantage for marine scientists.
\item Examples of shark fins.
\end{itemize}
\end{frame}


\begin{frame}
\frametitle{Examples of shark fin images showing the unique dorsal part}
\begin{figure}
\centering
\mbox{\subfigure[Shark fin 1]{\includegraphics[width=1in]{haai1.jpg}} \quad
\subfigure[Shark fin 2]{\includegraphics[width=1in]{haai2.jpg}}}
\end{figure}
\begin{figure}
\centering
\mbox{\subfigure[Shark fin 3]{\includegraphics[width=1in]{haai3.jpg}} \quad
\subfigure[Shark fin 4]{\includegraphics[width=1in]{haaim.png}}}
\end{figure}
\end{frame}


\begin{frame}
\frametitle{Background}
\begin{itemize}
\item Basic principle of segmentation algorithms.
\begin{itemize}
\myitem Partitioning a digital image into multiple segments to locate an object or
boundaries in the image.
\end{itemize}
\item Applications.
\begin{itemize}
\myitem Medical imaging to locate tumours.
\begin{figure}
 \centering
 \includegraphics[width=1.7in, height=0.8in]{braintumor.jpg}
 \caption{Braintumours detected by means of image segmentation}
 \end{figure}
\myitem Face and fingerprint recognition.
\myitem Video surveillance.
\end{itemize}
\item Many different techniques.
\end{itemize}
\end{frame}

\begin{frame}
\frametitle{Algorithms from the scikit-image Python library}
\begin{itemize}
\item The Watershed\cite{scikit-image} algorithm.
\begin{itemize}
\myitem Method?
\myitem Challenge?
\end{itemize}
\begin{figure}
 \centering
 \includegraphics[width=2in]{watershed.png}
 \caption{The Watershed algorithm}
 \end{figure}

\item The Random Walker\cite{scikit-image} algorithm.
\begin{itemize}
\myitem Method?
\myitem Challenge?
\end{itemize}
\end{itemize}
\end{frame}

\begin{frame}
\frametitle{The region growing GrowCut segmentation algorithm}
\begin{itemize}
\item Dr. Nathan Faggian\cite{growcut}, Senior Technical Officer at Bureau
of Meteorology, Australia.
\item How does it work? 
\begin{itemize}
\myitem Cellular Automata\cite{cellularoutomata}.
\myitem
\myitem Attacking strategies/evolution rule.
\end{itemize}
\begin{figure}
\centering
\includegraphics[width=1.7in, height=0.8in]{haaim.png}
\caption{The GrowCut algorithm}
\end{figure}
\item
\end{itemize}
\end{frame}


\begin{frame}
\frametitle{DARWIN -- Alternative software}
\begin{itemize}
\item DARWIN\cite{Darwin} is open source software, allowing marine scientists to
maintain information for the study of various behavioural and ecological
patterns of bottlenose dolphins.
\item The software also makes use of images of the dorsal fin of the dolphin, on
which segmentation then takes place.
\end{itemize}
\begin{figure}
 \centering
 \includegraphics[width=1.5in, height=0.9in]{Darwin.jpg}
 \caption{User interface}
\end{figure}
\end{frame}

\begin{frame}
\begin{itemize}
\item An article\cite{Darwin} appeared in Die Burger claiming that this
software could also be used to estimate the size of the Great White shark
population in the Gansbaai area by means of shark fin identification and
matching.
\item The validity of this claim was tested by Ms. Andreotti, which found it to
be unreliable in 54\% of the cases.
\item Motivation.
\end{itemize}
\end{frame}


\begin{frame}
\frametitle{Work in progress} 
\begin{itemize}
\item We found that the GrowCut algorithm performed the best on our data and
focused our attention on that.  Also because of its simplicity and easy
implementation in a language such as Python.
\item Putting together a pipeline for Ms. Andreotti, in collaboration with Tessa
Marais, a MsC student in Applied Mathematics, Stellenbosch University.
\item Current challenges and possible solutions.
\begin{itemize}
\myitem Specification of the coordinates and setting the orientation.
\end{itemize}
\item Future work.
\begin{itemize}
\myitem Damping function and attacking strategies.
\end{itemize}
\end{itemize}
\end{frame}


\begin{frame}
\frametitle{Conclusions}
\begin{itemize}
\item Joy and satisfaction of a project of this degree.
\item Demo of the GrowCut algorithm.
\end{itemize}
\end{frame}


\begin{frame}
\frametitle{Bibliography}
\bibliographystyle{plain}
\bibliography{ProjectProposalPresentation}
\end{frame}


\end{document}
